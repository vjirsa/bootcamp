
%\setcounter{page}{43}

\section{Complex Number Theory}
There are polynomials, such as $f(x)=x^{2}+1$, which do not have a root, $f(x)=0$ for
$x \in \mathbb{R}$. This is one of the reasons to extend the numbers from
$x \in \mathbb{R}$ to $z \in \mathbb{C}$ where $\mathbb{C}$ is the set of complex numbers.
Beyond this algebraic motivation there are many applications of complex numbers.

\subsection{Representations and Basic Properties}
A complex number $z$ consists of a pair of real numbers called the real and imaginary part,
respectively, and a "new" number '$i$' which has the property $i^2=-1$. While the real
numbers can be represented as points on a line, a complex number is given as a point in a
plane (called the complex plane) where the coordinates are its real (horizontal axis) and
imaginary (vertical axis) part.
\bnn z=\underbrace{\, \overbrace{a}^{\mbox{\small real}} \,
+ \;\; i \, \overbrace{b}^{\mbox{\small imaginary}}}_{\mbox{\small complex number}}
\qquad \mbox{with}  \quad a, b \in \mathbb{R} \enn
\begin{figure}[!h]
    \centerline{\epsfxsize=10cm  \epsfbox{matlab/fig39.eps}} \svs
    \caption{Representation of a complex number} \label{fig49}
\end{figure} \vs

The real and imaginary part, $a$ and $b$, can be expressed by a distance from the origin $r$
and an angle $\varphi$ (remember polar coordinates) in terms of $a=r\cos\varphi$ and
$b=r\sin\varphi$ which leads to
\bnn  z=a + i\, b = r\, \cos\varphi + i\,r\,\sin\varphi=r\,e^{i\,\varphi} \qquad \mbox{with} \quad
r=\sqrt{a^2+b^2} \quad \varphi=\arctan\frac{b}{a} \enn

\newpage
{\bf Remarks:}
\begin{itemize}
\item We will proof later by Taylor expansion that the relation $\cos\varphi+i\,\sin\varphi=e^{i\,\varphi}$ is true.
\item All basic computations such as addition, subtraction, multiplication or division are defined for $z \in \mathbb{C}$.
\item The inverse relation $\varphi=\arctan\frac{b}{a}$ is not unique because $\frac{-b}{-a}=\frac{b}{a}$. However, the
first number is the third quadrant whereas the second number is in the first quadrant.
Most computer languages, therefore have a second
function to calculate the inverse tangent (usually called atan2 or so) which accepts two arguments, i.e. $b$ and $a$
and not only their ratio $\frac{b}{a}$ and returns the correct angle in the $[0, 2\pi]$ or $[-\pi, \pi]$ range.
\item The number $i$ is called the imaginary unit and is defined as $i^2=-1$. It represents a very powerful tool to
simplify calculations, in particular when trigonometric functions are involved. From its definition we find readily
$i=\pm \sqrt{-1}, \;\; i^3=-i, \;\; \mbox{and} \;\; i^4=1$.
\end{itemize}  \svs


{\bf Rules for dealing with complex numbers:} $\qquad z=a + i\, b = r\, e^{i\, \varphi}$
\bnn \begin{array}{crcl} \svs
\mbox{Addition:} & \quad (a_1 + i\, b_1) + (a_2 + i\, b_2) & = & a_1 + a_2 + i\,(b_1+b_2) \\  \svs
\mbox{Multiplication:} & \quad (a_1 + i\, b_1) \; (a_2 + i\, b_2) & = & a_1\,a_2-b_1\,b_2+i\,(a_1\,b_2+a_2\,b_1) \\
       \quad & r_1\, e^{i\,\varphi_1} \; r_2\, e^{i\,\varphi_2} & = & r_1\, r_2\,e^{i\,(\varphi_1+\varphi_2)}
\end{array} \enn

\begin{figure}[!h]
    \centerline{\epsfxsize=10cm  \epsfbox{matlab/fig40.eps}} \svs
    \caption{Adding two complex numbers}  \label{fig50}
\end{figure} \vs

\begin{center}
$\Rightarrow \quad$ Addition of two complex numbers is done by adding their corresponding vectors \\ \svs
$\Rightarrow \quad$ Multiplication of two complex numbers results in the product of the individual \\
                          amplitudes and the sum of the phases
\end{center}

\centerline{\bf All the properties of real numbers are still preserved!} \vs

{\bf Examples:} $\qquad z_1=1+2\,i \qquad z_2=2-\,i$
\bnn z_1+z_2=1+2+i\,(2-1)=3+i \qquad\quad z_1\, z_2=1\cdot2-2\cdot(-1)+i\,(1\cdot(-1)+2\cdot2)=4+3\,i \enn

\subsection{Complex conjugate}
The complex conjugate of $z=a+i\,b$ is defined as $z^*=a-i\,b$
\bnn z^*=a+i\,b=r\,e^{-i\,\varphi} \qquad\quad z\,z^*=(a+i\,b)\,(a-i\,b)=a^2+b^2=\abs{z}^2 \neq z^2 \enn
\begin{figure}[!h]
    \centerline{\epsfxsize=9.5cm  \epsfbox{matlab/fig41.eps}} \svs
    \caption{The complex number $z$ and its complex conjugate $z^*$}  \label{fig51}
\end{figure}

Compare the following:
\bnn \begin{array}{ccccccccccc} \svs
z^2 &=& z\,z &=& (a+i\,b)^2 &=& a^2-b^2+2\,i\,a\,b &=& (r\,e^{i\,\varphi})^2 &=& r^2\,e^{2\,i\,\varphi} \\
\abs{z}^2 &=& z\,z^* &=& (a+i\,b)\,(a-i\,b) &=& a^2+b^2 &=& r\,e^{i\,\varphi}\;r\,e^{-i\,\varphi} &=& r^2
\end{array} \enn \svs

{\bf Some more rules:}

\bnn \mbox{Complex division:} \quad \frac{z_1}{z_2}=\frac{a_1+i\,b_1}{a_2+i\,b_2}=\frac{z_1\,z_2^*}{z_2\,z_2^*}
        =\frac{z_1\,z_2^*}{\abs{z_2}^2}=\frac{a_1\,a_2+b_1\,b_2+i\,(a_2\,b_1-a_1\,b_2)}{a_2^2+b_2^2} \enn \svs
\bnn {\cal R}(z)=\frac{1}{2}(z+z^*)=\frac{1}{2}(a+i\,b+a-i\,b)=a \qquad\quad
        {\cal I}(z)=\frac{1}{2}(z-z^*)=\frac{1}{2}(a+i\,b-(a-i\,b))=b  \enn \svs
\bnn
    e^{i\,\varphi}=\cos\varphi+i\,\sin\varphi \qquad \Rightarrow \quad
    \cos\varphi=\frac{1}{2}(e^{i\,\varphi}+e^{-i\,\varphi}) \qquad
    \sin\varphi=\frac{1}{2\,i}(e^{i\,\varphi}-e^{-i\,\varphi})
\enn
